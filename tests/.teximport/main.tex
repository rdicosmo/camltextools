% Document Type: LaTeX
% Master File: tests.tex
\documentclass{article}
\usepackage{a4}
\usepackage{alltt}
\usepackage{graphicx}
\usepackage{epic,eepic}
\usepackage{amssymb}

\newcommand{\run}{{\mathbb R}}
\newcommand{\nun}{{\mathbb N}}
\newcommand{\bral}{\left<}
\newcommand{\brar}{\right>}
\newcommand{\Th}[1]{\mathcal{T}^{h}_{#1}}
\newcommand{\Sh}[1]{\mathcal{S}^{h}_{#1}}
\newcommand{\Nui}[1]{{\mathcal N}_{#1}}
\newcommand{\ovl}{\overline}
\newcommand{\dps}{\displaystyle}
\newcommand{\Hdiv}[1]{{\cal H}(\mbox{div},#1)}
\newcommand{\p}{\partial}
\newcommand{\nab}{\mbox{\boldmath $\nabla$}}
\newcommand{\ocamlpiiil}{{\sf OcamlP3l}}
\newcommand{\ocaml}{{\sf Ocaml}}
\newcommand{\pppl}{{\sf p3l}}

\title{Coupling numerical codes using the \ocamlpiiil\ functional parallel programming system
       \thanks{The OcamlP3l system has been partially funded by a Galileo bilateral France-Italy project. }}
\author{Fran\c{c}ois Cl\'ement, Roberto Di Cosmo, Vincent Martin, Pierre Weis, Arnaud Vodicka}

\begin{document}
\maketitle

\begin{abstract}
  Writing parallel programs is not easy, and debugging them is usually
  a nightmare.  To cope with these difficulties, a structured approach
  to parallel programs using skeletons and template based compiler
  techniques has been developed over the past years by several
  researchers, including the P3L group in Pisa. The \ocamlpiiil\ 
  system marries the Ocaml functional programming language with the
  P3L skeletons, yielding a powerful parallel programming environment,
  that has evolved over the years.

  In this paper, we report on the successful application of the \ocamlpiiil\
  in the field of scientific computing, where the system has been used to 
  solve the problem of coupling numerical codes, obtaining parallelization
  for free.

  The interaction has been extremely fruitful, as, in the process of solving the
  coupling problem, a wealth of new ideas have emerged on the design of the
  system, which are incorporated in the current version of  \ocamlpiiil.

\end{abstract}

\section{Introduction and Overview}

In a skeleton based parallel programming model 
\cite{cole-th,ic-parle-93-1,fgcs-firenze} a set of \textit{skeletons},
i.e. of second order functionals modelling common parallelism
exploitation patterns are provided to the user/programmer. The
programmer must use the skeletons to give parallel structure to
an application and  uses a plain sequential language to
express the sequential portions of the parallel application as
parameters to the skeletons. He/she has no other way to express
parallel activities but skeletons: no explicit process creation,
scheduling, termination, no communication promimitives, no shared
memory, no notion of being executing a program onto a parallel
architecture at all.

\ocamlpiiil\ is a programming environment that allows to write parallel
programs in \ocaml\footnote{See URL
  \texttt{http://pauillac.inria.fr/ocaml/}} according to the skeleton
model supported by the parallel language \pppl\footnote{See URL
  \texttt{http://www.di.unipi.it/.susanna/p3l.html}}, provides
seamless integration of parallel programming and functional
programming and advanced features like sequential logical debugging
(i.e. functional debugging of a parallel program via execution of the
architecture at allparallel code onto a
sequential machine) of parallel programs and strong typing, useful
both in teaching parallel programming and in building of
full-scale applications\footnote{See URL
  \texttt{http://qui.di.unipi.it/ocamlp3l.html} you will find relevant
  information, up to date references, documentation, examples,
  distribution code and dynamic web pages showcasing the \ocamlpiiil\
  features.}.


%
% partie \ocamlpiiil
%
In \ocamlpiiil, as described in the seminal paper~\cite{Ocamlp3lMlw98},
we provide, for any user program, 3 possible semantics: 
\begin{description}
  \item[sequential] the user program is linked against a sequential implementation
                    of the skeletons, so the resulting executable can be run on
                    a single machine, as a single process, and easily debugged using
                    standard debugging techniques and tools for sequential programs
  \item[parallel]  the user program is linked against a parallel implementation
                   of the skeletons, and the resulting executable is a generic
                   SPMD program that can be deployed on a parallel machine, a cluster,
                   or a network of workstations
  \item[graphical] the user program is linked against a graphical implementation of
                   the skeletons, so that the resulting programs, when executed, 
                   display a picture of the parallel computational network that is
                   deployed when running the parallel implementation.
\end{description}
Of course, our goal is to guarantee that the sequential and the parallel execution of
any user program do produce the same results.

\section{Skeletons as parametric stream processors}

In \ocamlpiiil, skeletons are \emph{compositional}, and, in the parallel implementation, 
a  skeleton is clearly realized as a \emph{stream processor},
i.e.\  a function transforming the input stream of incoming data into an output stream of
outgoing data, and is parameterized by some other functions and/or stream processors.

Of course, we want this fact to be apparent in the sequential implementation as well,
so the type of skeletons is that of stream processors, and we provide an abstract type
\verb|'a stream| of streams, on which the sequential implementation of the skeletons is
built.

\subsection{The skeleton combinators in OcamlP3l}
The basic building blocks of the skeleton language 
implemented within this current release of \ocamlpiiil\ are basically of four kinds:
\begin{itemize}
\item \textit{task parallel} skeletons, modelling parallelism
  exploited between \textit{independent} processing activities
  relative to different input data. In this set we have: pipe
  and farm, that correspond to
  the usual task parallel skeletons appearing both in \pppl\ 
  and in other skeleton models
  \cite{cole-th,ic-parle-93-1, darli-to-1}.
\item \textit{data parallel} skeletons, modelling parallelism exploited
  computing different parts of the same input
  data. In this set, we provide \verb|mapvector| and
  \verb|reducevector|. 
  Such skeletons are not yet as powerful as the
  \verb|map| and \verb|reduce| skeletons of \pppl. Instead, they are quite similar to
  the map($*$) and reduce ($/$) functionals of the Bird-Meertens formalism
  discussed in \cite{bird1} and the \textsf{map} and \textsf{fold} skeletons in SCL
\cite{darli-to-1}. The \texttt{mapvector} skeleton models the parallel application of
a generic function $f$ to all the items of a vector data structure,
whereas the reducevector skeleton models a parallel computation folding all
the elements of a vector with a commutative and associative binary
operator $\oplus$). 
\item a \textit{control} skeleton, \verb|loop| which is not parallel \textit{per se},
      but is necessary to iterate the execution of other skeletons 

\item two \textit{interface} skeletons, that allow to freely move back and forth
      bewteen the parallel and sequential worlds: \verb|seq| converts a sequential
      function into a one node parallel computation network,
      and \verb|parfun| converts a parallel computation network into a
      stream processing function

\item a \textit{parallel execution scope delimiter}, given by the function \verb|pardo| that
      must contain all code that invokes a \verb|parfun| 
\end{itemize}

\subsection{Skeleton syntax, semantics and types}

We briefly describe here the syntax, informal semantics and types for
each of the skeletons in the system. First of all, let's discuss the fact that the
actual types used in the sequential implementation of the skeletons are
polluted by a certain amount of \verb|unit| types, with respect to the types
one would expect. We can discuss the reason of this choice by examining the case
of the simplest skeleton, \verb|seq|, that allows the encapsulation of an \ocaml\ function $f$ into
a sequential process which applies $f$ to all the inputs received in
the output stream. Any \ocaml\ function with type 
\begin{center}
  \verb1 f: 'a -> 'b 1
\end{center}
\noindent can be encapsulated in an instance of the seq
skeleton as follows:
\begin{center}
  \verb|seq(f)|
\end{center}
and we would expect then \verb|seq| to have the type \verb|('a -> 'b) -> 'a stream -> 'b stream|
but in the library, we find \verb|(unit -> 'a -> 'b) -> unit -> 'a stream -> 'b stream|
instead. This is due to the fact that, in real-world application, the user functions may
need to hold a sizeable amount of local data, like the huge matrices that are initialised in
the numerical application described further on, and we need to allow the user to tell whether
\begin{itemize}
  \item this data is initialised \emph{globally} once and then replicated in every copy of the stream processor that
        may be performed by a \verb|farm| or a \verb|mapvector| skeleton; this was the case of
	the previous versions of \ocamlpiiil, where one could write
\begin{alltt}
let f = let localdata = do_huge_intialisation_step ()
        in fun x -> compute (localdata,x);;
\ldots{}
farm(seq(f),10)
\end{alltt}
%
  \item this data is initialised \emph{locally} by each stream processor \emph{after} the
        copy performed by a \verb|farm| or a \verb|mapvector| skeleton; this was not possible
        in the previous versions of \ocamlpiiil, but it is easily achievable with the new types
%
\begin{alltt}
let f () = let localdata = do_huge_intialisation_step ()
           in fun x -> compute (localdata,x);;
\ldots{}
farm(seq(f),10)
\end{alltt}
%
        now, when the \verb|farm| skelton creates $10$ copies of \verb|seq(f)|, each copy is
        created by passing $()$ to \verb|seq|, which in turn passes $()$ to $f$, producing
        the allocation of a different copy of $localdata$ for each instance\footnote{The initialization step may do weird, non referential transparent things, like opening file descriptors or network connections to other services, that we
do not want to be shared at all by the different instances of the user function.}.
        Notice that the old behaviour, namely, a global initialisaztion shared among all copies, is
        still easily achievable by writing
%
\begin{alltt}
let f = let localdata = do_huge_intialisation_step ()
        in fun () -> fun x -> compute (localdata,x);;
\ldots{}
farm(seq(f),10)
\end{alltt}
%
\end{itemize}

To sum up, the extra \verb|unit| parameters give the programmer the possibility to choose whether local
initialisation data in his user functions is shared among all copies or not. In other words, we can regard
the skeleton combinators in the current version of \ocamlpiiil\ as ``delayed skeletons'', or ``skeleton factories'',
producing\emph{an instance} of a skeleton every time they are passed an argument $()$ of type \verb|unit|.

We can now detail the other skeletons:

\begin{description}
\item[Farm] The farm skeleton, written \texttt{farm},
  computes in parallel a function $f$ over different data items
  appearing onto its input stream.  From a functional viewpoint, given
  a stream of data items $x_1, \ldots , x_n$, $\texttt{farm}(F,k)$
  computes $f(x_1), \ldots, f(x_n)$ ($F$ being the skeleton
  parameter).  Parallelism is exploited by having $k$ independent,
  parallel processes computing $f$ on different items of the input
  stream. \\
  If $F$ has type \texttt{(unit -> 'b stream -> 'c stream)}, and $n:\texttt{int}$, then 
  $farm(F,n)$ has type \texttt{unit -> 'b stream -> 'c stream}

\item[Pipeline skeleton] The pipeline skeleton, denoted by the infix
  operator \verb1|||1, performs in parallel the computations relative
  to different stages of a function over different data items of the
  input stream. Functionally, $F_1 \verb1|||1 F_2 \ldots \verb1|||1
  F_N$ computes $f_n(\ldots f_2(f_1(x_i))\ldots)$ over all the data
  items $x_i$ belonging to the input stream. Parallelism is exploited
  by having $n$ independent parallel processes. Each process computes
  a function $f_i$ over the data items produced by process computing
  $f_{i-1}$ and delivers results to process computing $f_{i+1}$.
  If $F_1$ has type \texttt{(unit -> 'a stream -> 'b stream)}, and $F_1$ has type \texttt{(unit -> 'b stream -> 'c stream)},
  then $F_1|||F_2$ has type \texttt{unit -> 'a stream -> 'c stream}
\marginpar{Pierre: tu veux mettre c,a en forme de regles d'inference?}
\item[Map skeleton] The map skeleton, written \texttt{mapvector},
  computes in parallel a function over all the data items of a vector,
  generating a new vector. Therefore, for each vector $X$ appearing
  onto the input data stream, $\texttt{mapvector}(F,n)$ computes the
  function $f$ over all the items of the vector, using $n$ different,
  parallel processes computing $f$ over distinct vector items.

  If $F$ has type \texttt{(unit -> 'a stream -> 'b stream)}, and $n:\texttt{int}$, then 
  $mapvector(F,n)$ has type \texttt{unit -> 'a array stream -> 'b array stream}

\item[Reduce skeleton] The reduce skeleton, denoted by the keyword
  \texttt{reducevector}, folds a function over all the data items of a
  vector. Therefore, $\texttt{reducevector}(F,n)$ computes $ x_1 f x_2
  f \ldots f x_n$ out of the vector $x_1, \ldots ,x_n$, for each one
  of the vectors appearing onto the input data stream. The computation
  is performed using $n$ different, parallel processes computing $f$.

  If $F$ has type \texttt{(unit -> 'a * 'a stream -> 'a stream)}, and $n:\texttt{int}$, then 
  $reducevector(F,n)$ has type \texttt{unit -> 'a array stream -> 'a stream}
\end{description}

\begin{figure}[htbf]
\begin{alltt}
val seq : ?col:int -> (unit -> 'a -> 'b) -> unit -> 'a stream -> 'b stream
val ( ||| ) :
  (unit -> 'a stream -> 'b stream) ->
  (unit -> 'b stream -> 'c stream) -> unit -> 'a stream -> 'c stream
val loop :
  ?col:int ->
  ('a -> bool) * (unit -> 'a stream -> 'a stream) ->
  unit -> 'a stream -> 'a stream
val farm :
  ?col:int ->
  ?colv:'a list ->
  (unit -> 'b stream -> 'c stream) * int ->
  unit -> 'b stream -> 'c stream
val mapvector :
  ?col:int ->
  ?colv:'a list ->
  (unit -> 'b stream -> 'c stream) * int ->
  unit -> 'b array stream -> 'c array stream
val reducevector :
  ?col:int ->
  ?colv:'a list ->
  (unit -> ('b * 'b) stream -> 'b stream) * int ->
  unit -> 'b array stream -> 'b stream
\end{alltt}
\caption{The types of the \ocamlpiiil\ skeleton combinators }
\end{figure}

\subsection{ The  {\tt parfun} construction }

\begin{alltt}
val parfun :
  (unit -> unit -> 'a stream -> 'b stream) -> 'a stream -> 'b stream
val pardo : (unit -> 'a) -> 'a
\end{alltt}

\subsection{ Structure of the program }\label{f11}
\subsection{ Balancing load: colors }

\section{Using the system}


%
% partie numerique
%
\section{The coupling problem}
%
% description of the numerical problem
% on explique que le parallelisme n'etait pas un but primaire de la manip

Let $\Omega$ be a convex domain in $\run^d,\,\, d=2$ or $3$, and we denote by
$\Gamma=\partial \Omega$ the boundary of $\Omega$. We suppose that the flow in
$\Omega$ is governed by a conservation equation together with Darcy's law
relating the gradient of the pressure $p$ to the Darcy velocity ${\bf u}$:
\begin{equation}\label{prob}\begin{array}{rllll}
\div\,{\bf u} &=& q &\qquad\mbox{in }\Omega\\
{\bf u} &=& -{\bf K }\nab\,p   &\qquad\mbox{in }\Omega\\
p&=&\ovl{p}&\qquad\mbox{on }\Gamma,
\end{array}\end{equation}
where $p$ is the pressure, 
      ${\bf u}$ the Darcy velocity, 
      ${\bf K}$ the hydraulic conductivity (or permeability) tensor,
      $q$ a source term and
      $\overline{p}$ the given pressure on the boundary $\Gamma$.
We suppose that ${\bf K}$ is a symmetric, positive definite and uniformly
bounded tensor.

Let $\Omega$ be decomposed into $n$ non-overlapping subdomains, also called
blocks, $\Omega_i, i\in I=\{1,2,\ldots,n\}$, and we denote by $\Gamma_{i}$ the
part of the boundary of $\Omega_{i},\, i \in I$ in common with the boundary of
$\Omega$. Hence,
\[
\bar\Omega = \bigcup_{i\in I} \bar\Omega_i, 
\quad \mbox{ and } \quad 
\Gamma_{i} = \partial\Omega_{i}\cap\Gamma,\quad i \in I = \{1,2,\ldots,n\},
\]
where $\bar\Omega_i$ is the closure of the open set $\Omega_i$.
Let $\Sigma_{i}$ be the geometric {\it internal boundary} of $\Omega_i$, i.e.
the interior of the internal boundary of $\Omega_i$, and $\Sigma$ the union of
all internal boundaries.
\[
\Sigma_{i} = \stackrel{\circ}{\ovl{(\p \Omega_i \backslash \p \Omega)}},\quad i
\in I, \qquad \Sigma = \bigcup_{i \in I} \Sigma_{i}.
\]
Then we can define the geometric {\it interface between subdomains} $\Omega_i$ and $\Omega_j$ as 
\[
\Sigma_{ij} = \Sigma_{ji} = \Sigma_{i} \cap  \Sigma_{j}, \qquad \Sigma_{ii} = \emptyset,
\quad i, j \in I.
\]
If $\Sigma_{ij} \neq \emptyset$, the subdomains $\Omega_i$ and $\Omega_j, \ 
i \neq j \in I$, are called neighbors. 

Let $\nu_i$ denote the external unit normal to $\Omega_i, i \in I$. We denote
the $L^2(\Omega_i)$ or $(L^2(\Omega_i))^d$ the inner product by $(\cdot,
\cdot)_{i}$. In these notations, we omit the subscript $i$ to refer to the
entire domain $\Omega$. Let the interface inner product on $L^2(\Sigma_i)$ be
denoted by $\bral\cdot, \cdot \brar_i$ and on $L^2(\Sigma_{ij})$ by $\bral
\cdot, \cdot \brar_{ij}$. Note that if $\Sigma_{ij}=\emptyset$, this inner
product is zero.

We denote by $p_i,{\bf u}_i,{\bf K}_i,$ and $q_i$ the restrictions of $p,{\bf
  u},{\bf K},$ and $q$ respectively to $\Omega_{i},\, i \in I$, and by
$\ovl{p}_i$ the restriction of $\ovl{p}$ to $\Gamma_i,\, i \in I$.  We can
rewrite the above problem (\ref{prob}) as a transmission or multiblock problem:
\begin{equation}\label{probt}\begin{array}{rllll}
\div \,{\bf u}_i &=& q_i &\qquad\mbox{in }\Omega_{i},&i \in I,\\
{\bf u}_i &=& -{\bf K}_i\nab\,p_i &\qquad\mbox{in }\Omega_{i},&i \in I,\\
p_i &=& \ovl{p}_i &\qquad\mbox{on }\Gamma_{i},&i \in I,
\end{array}\end{equation}
together with transmission conditions
\begin{equation}\label{interftrans}\begin{array}{rllll}
p_i &=& p_j &\qquad\mbox{on }\Sigma_{ij},&i,j \in I,\\
{\bf u_i \cdot  \nu_i} &=& - {\bf u_j \cdot \nu_j}
          &\qquad\mbox{on }\Sigma_{ij},&i,j \in I.\\ 
\end{array}\end{equation}

The two equations in (\ref{interftrans}) yield the continuity of pressure and of normal 
velocity (as $\nu_i = -\nu_j$) across each interface $\Sigma_{ij} (\neq \emptyset)$ 
between two neighboring subdomains 
$\Omega_i$ and $\Omega_j, \ i,j \in I$. These interface equations can be rewritten 
with Robin conditions. 

Let $\alpha_{ij} > 0$ and $\alpha_{ji} > 0$ be the
Robin coefficients on the interface $\Sigma_{ij} = \Sigma_{ji} , \ i,j \in I$. 
First, multiply the first equation by $\alpha_{ij}$ and substract the second equation. 
Second, multiply the first equation by $\alpha_{ji}$ and add the second equation. 
Thus the two equations in (\ref{interftrans}) are equivalent to the Robin
interface conditions 

\begin{equation}\label{interfrob}\begin{array}{rllll}
- {\bf u_i \cdot \nu_i}  + \alpha_{ij} p_i 
&=& 
{\bf u_j \cdot  \nu_j} + \alpha_{ij} p_j 
&\qquad\mbox{on }\Sigma_{ij},&i,j \in I,\\
- {\bf u_j \cdot  \nu_j} + \alpha_{ji} p_j 
&=& 
{\bf u_i \cdot  \nu_i}  + \alpha_{ji} p_i 
&\qquad\mbox{on }\Sigma_{ij},&i,j \in I.\\
\end{array}\end{equation}

Clearly the solution to (\ref{prob}) satisfies (\ref{probt}) and (\ref{interftrans}) or 
equivalently (\ref{probt}) and (\ref{interfrob}). Conversely, by elliptic regularity,
the solution to (\ref{probt}) and (\ref{interfrob}) satisfies (\ref{prob}), so the two 
formulations are equivalent.

\subsection{Weak multiblock problem}
Throughout the rest of this section, for simplicity, we will assume that
homogeneous Dirichlet boundary conditions are imposed on $\p \Omega$: 
$\ovl{p} = 0 \mbox{ on }\Gamma$.

Let us define the approximation spaces used in this paper.
\begin{equation}\begin{array}{lll}
  {\bf Z_i} = \Hdiv{\Omega_i},&i \in I. & 
{\bf Z} = \dps \bigoplus_{i \in I}  {\bf  Z_i} \not \subset \Hdiv{\Omega} . \\
  N_i = L^2(\Omega_i),&i \in I. & N = \dps \bigoplus_{i \in I} N_i =  L^2(\Omega). \\
  \Lambda_i = L^2(\Sigma_i),&i \in I. & \Lambda = \dps \bigoplus_{i \in I} \Lambda_i. \\
\end{array}\end{equation}

With these notations, we can write the mixed formulation. A weak solution 
of (\ref{probt})-(\ref{interfrob}) 
is a couple $({\bf u}, p) \in  {\bf Z} \times N$ such that for each 
$i \in I$, $p_i|_{\Sigma_i} \in \Lambda_i$, and
${\bf u_i \cdot \nu_i}|_{\Sigma_i}  \in \Lambda_i$, and 
 \begin{equation}\label{probtw}\begin{array}{rcll}
\left( K^{-1}{\bf u},\, {\bf v} \right)_{i} - 
\left( \div\; {\bf v}, \, p \right)_{i}  & = & \dps 
- \bral {\bf v \cdot  \nu_i} , \, p_{i} \brar_{i},  & {\bf v} \in {\bf Z_{i}}, \\
\left( \div\; {\bf u}, \, r \right)_{i} &= &
 \left(q_i, \, r \right)_{i},  & r \in N_{i},\\
\dps \sum_{j \in I} \bral - {\bf u_{i} \cdot  \nu_i}  + \alpha_{ij} p_i  , 
\, \mu_{i} \brar_{ij} & = & 
\dps \sum_{j \in I} \bral {\bf u_{j} \cdot  \nu_j}  + \alpha_{ij} p_j  , 
\, \mu_{i} \brar_{ij},  & \mu_{i} \in \Lambda_{i}.\\
\end{array}\end{equation}

\subsection{Meshes on the subdomains}
Let $\Th{i}$ be a conforming finite element partition of the subdomain
$\Omega_i,\ i \in I$, where the maximal element diameter is bounded by the real
$h>0$, and the interior of each element face (or edge in 2D) are entirely
contained in $\Sigma_i$ or in $\Gamma_i = \p \Omega_i \cap \p \Omega$.  The
meshes of two neighboring subdomains need not match at their interface
$\Sigma_{ij}$.

Let us now define more precisely the interface meshes. We have seen that two
subdomains are called {\it neighbors} if their interface $\Sigma_{ij} \neq
\emptyset$. For $i \in I$, we denote the number of neighbors of a subdomain
$\Omega_i$ by $n_i$, and we denote the set of indices of these neighbors by
$\Nui{i} = \{j_1, j_2, \ldots, j_{n_i}\}$. The sets $I$ and $\Nui{i}$ inherit
the natural order on $\nun$. Thus $k \in {1,2,\ldots, n_i} \mapsto j_k \in
\Nui{i}$ is increasing.  For an $i \in I$ and a $j \in \Nui{i}$, the trace on
$\Sigma_{ij}$ of the mesh $\Th{i}$ of $\Omega_{i}$ is denoted by $\Sh{ij}$.  In
the same way, for a $j \in I$ and an $i \in \Nui{j}$, the trace on $\Sigma_{ji}
= \Sigma_{ij}$ of the mesh $\Th{j}$ is denoted by $\Sh{ji}$.  In general
$\Sh{ji} \neq \Sh{ij}, i \neq j$, as the meshes $\Th{i}$ and $\Th{j}$ do not
match. (See Figure~\ref{f:dom})

\begin{figure} [htbp]
  \begin{center}
\setlength{\unitlength}{1.5mm}

\begin{picture}(90,50)(0,0)
  \allinethickness{0.5mm}
  \qbezier(15,25)(25,32.5)(45,32.5)
  \qbezier(45,32.5)(57.5,32.5)(65,40)
  \drawline(65,40)(70,25)(60,15)
  \dashline[+50]{2}(60,15)(50,10)(35,10)(25,15)
  \drawline(25,15)(15,25)

  \allinethickness{0.2mm}
  \dashline[+50]{2}(70,45)(66,41)(71,25)(80,20)
  \dashline[+50]{2}(80,19)(71,24)(61,14)(65,5)
  \dashline[+50]{2}(20,5)(24,14)(14,24)(0,20)

  \put(45,22){\makebox(0,0){$\Omega_i$}}
  \put(35,35){\makebox(0,0){$\Gamma_i$}}
  \put(65,30){\makebox(0,0){$\Sh{ij_1}$}}
  \put(65,20){\makebox(0,0)[br]{$\Sh{ij_2}$}}
  \put(20,20){\makebox(0,0)[bl]{$\Sh{ij_{n_i}}$}}
  \put(80,35){\makebox(0,0){$\Omega_{j_1}$}}
  \put(75,10){\makebox(0,0){$\Omega_{j_2}$}}
  \put(10,10){\makebox(0,0){$\Omega_{j_{n_i}}$}}
  \put(69,32){\makebox(0,0)[bl]{$\Sh{j_1i}$}}
  \put(66,19){\makebox(0,0)[tl]{$\Sh{j_2i}$}}
  \put(19,19){\makebox(0,0)[tr]{$\Sh{j_{n_i}i}$}}
\end{picture}
\end{center}
  \caption{Subdomain~$\Omega_i$, its neighbors and the interface meshes.}
  \label{f:dom}
\end{figure}


We denote by $\Sh{i}$ the trace on $\Sigma_{i}$ of the mesh $\Th{i}$ of
$\Omega_i$.  It is also the union of all the interface meshes coming from the
subdomain $\Omega_i$. We call the mesh $\Sh{i}$ the {\it inner interface mesh}
of $\Omega_i$.

\[
\forall i \in I,\ \forall j \in \Nui{i} \quad \Sh{ij} = \Th{i}|_{\Sigma_{ij}}. 
\qquad \Sh{i} = \biguplus_{j\in \Nui{i}} \Sh{ij}. 
\]

The set of the ordered couples of neighbors is denoted by $\Nui{} = \{ (i,j), i
\in I, j \in \Nui{i}\}$. It is ordered according to lexicographic order on
$\nun^2$. We denote by $s$ the permutation
\[
\forall (i,j) \in \Nui{}, \quad s(i,j) = (j,i) \in \Nui{}. 
\]

We call {\it inner interface structure} $\Sh{}$ of the domain $\Omega$ the
union of all inner interface meshes (each geometric interface is thus counted
twice). It is ordered by $\Nui{}$.
\[
\Sh{} = \biguplus_{i \in I} \Sh{i} = \biguplus_{i \in I}\biguplus_{j\in \Nui{i}} \Sh{ij}
= \biguplus_{(i,j) \in \Nui{}} \Sh{ij}.
\]

Up to now, we have defined the interface meshes that were seen from a subdomain
$\Omega_i$. Now, we define the interface meshes of the neighbors of $\Omega_i$.
For $(i,j) \in \Nui{}$, we denote $\tilde \Sh{ij} =\Sh{ji}$. We denote by
$\tilde \Sh{i}$ the {\it outer interface mesh} of the subdomain $\Omega_i$ that
is the union of the interface meshes that are the traces of the neighbors of
$\Omega_i$. It is also ordered by $\Nui{i}$.
\[
\tilde \Sh{i} = \biguplus_{j\in \Nui{i}} \tilde \Sh{ij} = \biguplus_{j\in \Nui{i}} \Sh{ji}. 
\]
In the same way, we denote by $\tilde\Sh{}$ the {\it outer interface structure}
of the domain $\Omega$ that is the union of all outer interface meshes. It is
also the union of all interface meshes, seen from a subdomain or its neighbor,
but this time it is ordered by $s(\Nui{})$,
\[
\tilde \Sh{} = \biguplus_{i \in I} \tilde\Sh{i}  
=\biguplus_{i \in I}\biguplus_{j\in \Nui{i}} \tilde\Sh{ij}
=\biguplus_{i \in I}\biguplus_{j\in \Nui{i}} \Sh{ji}
= \biguplus_{(i,j) \in s(\Nui{})} \Sh{ij}.
\]

\subsection{Discrete weak multiblock problem}
Let 
\[ {\bf Z_{h,i}} \times N_{h,i} \times \Lambda_{h,i} 
\subset  {\bf Z_{i}} \times N_{i} \times \Lambda_{i} , \quad i \in I,
\]
be the usual mixed finite element approximation made of Raviart-Thomas (and
Nedelec in 3D) spaces of lowest order (\cite{ravtho}, \cite{robtho},
\cite{ned86}), with the hybrid Lagrange multiplier spaces $\Lambda_{h,i}$ on 
$\Sigma_i$, \cite{brefor}. We recall that
$Z_{h,i}$ consists of some linear vector functions and that $N_{h,i}$ 
(respectively $\Lambda_{h,i}$) are 
made up of cellwise (resp. facewise in 3D or edgewise in 2D) constant scalar functions. 
Let 
\[ 
{\bf Z_{h}} = \bigoplus_{i \in I} {\bf Z_{h,i}}, \qquad 
N_{h} = \bigoplus_{i \in I}N_{h,i}, \qquad 
\Lambda_{h}= \bigoplus_{i \in I}\Lambda_{h,i}.
\]
We also recall that
\[  \div {\bf Z_{h,i}} = N_{h,i} \qquad \Lambda_{h,i} 
= {\bf Z_{h,i} \cdot \nu_{i}}|_{\Sigma_i} = P_0(\Sh{i}), \quad i \in I,
\]
where $P_0(\Sh{i})$ denotes the space of facewise (or edgewise in 2D) constant
functions on $\Sh{ij}$.

One can notice that the space $\Lambda_{h}$ contains two interface unknows on
each interface $\Sigma_{ij}, i \in I, j \in \Nui{i}$: the one that lives in
$P_0(\Sh{ij})$, and the other one that lives in $P_0(\Sh{ji})=
P_0(\tilde\Sh{ij})$.  As we did in the previous section for the interface
meshes, we introduce on the geometric interface between two subdomains
$\Sigma_{ij}$ two different approximation spaces $\Lambda_{h,ij}$ and $\tilde
\Lambda_{h,ij} =\Lambda_{h,ji}$.
\begin{equation}\begin{array}{rcl}
    \Lambda_{h,ij} & = & {\bf Z_{h,i} \cdot \nu_{i}}|_{\Sigma_{ij}} =
    P_0(\Sh{ij}), 
    \quad i \in I, \ j \in \Nui{i} \quad (\mbox{i.e. } (i,j) \in \Nui{}), \\ 
    \tilde\Lambda_{h,ij} =
    \Lambda_{h,ji} & = & {\bf Z_{h,j} \cdot \nu_{j}}|_{\Sigma_{ij}} =
    P_0(\Sh{ji}), \quad j \in I, \ i \in \Nui{j}.
\end{array}\end{equation}
Next we define the space $\tilde \Lambda_{h,i}$ that contains the functions on
the interface meshes $\tilde \Sh{i}$ of the neighbors of $\Omega_i,\ i \in I$.
This yields
\begin{equation}\begin{array}{rclcll}
\Lambda_{h,i} & = &  \dps \bigoplus_{j \in \Nui{i}}\Lambda_{h,ij} 
 & = &{\bf Z_{h,i} \cdot \nu_{i}}|_{\Sigma_i} 
= P_0(\Sh{i}), &\quad i \in I, \\
\tilde\Lambda_{h,i} & = & \dps \bigoplus_{j \in \Nui{i}} \tilde\Lambda_{h,ij} 
=  \bigoplus_{j \in \Nui{i}} \Lambda_{h,ji}
& = &{\bf Z_{h,j} \cdot \nu_{j}}|_{\Sigma_i} 
= P_0(\tilde \Sh{i})
, & \quad i \in I,
\end{array}\end{equation}
and
\begin{equation}\begin{array}{rcl}
\Lambda_{h} & = & \dps \bigoplus_{i \in I}\Lambda_{h,i}
= \bigoplus_{i \in I} \bigoplus_{j \in \Nui{i}} \Lambda_{h,ij} 
= \bigoplus_{(i,j) \in \Nui{} } \Lambda_{h,ji}, \\
\tilde\Lambda_{h} & = & \dps \bigoplus_{i \in I}\tilde\Lambda_{h,i}
= \bigoplus_{i \in I} \bigoplus_{j \in \Nui{i}} \tilde\Lambda_{h,ij} 
= \bigoplus_{i \in I} \bigoplus_{j \in \Nui{i}} \Lambda_{h,ji}
= \bigoplus_{(i,j) \in s(\Nui{}) } \Lambda_{h,ji}.
\end{array}\end{equation}

We write the discrete mixed finite element approximation for the original
problem (\ref{prob}) or, equivalently, for the transmission problem 
(\ref{probtw}) introducing interelement multipliers $\lambda_{h,i}$ 
(see \cite{MR95j:65161}, \cite{MR2001h:65140} and references therein). 

We seek $({\bf u_h}, p_h, \lambda_h) \in {\bf Z_h} \times N_h \times \Lambda_h$
such that
\begin{equation}\label{probtwd}\begin{array}{rcll}
\left( K^{-1}{\bf u_h},\, {\bf v} \right)_{i} - 
\left( \div\; {\bf v}, \, p_h \right)_{i}  & = & \dps 
- \bral {\bf v \cdot  \nu_i} , \, \lambda_{h,i} \brar_{i},  & {\bf v} \in {\bf Z_{h,i}}, \\
\left( \div\; {\bf u_h}, \, r \right)_{i} &= &
 \left(q_i, \, r \right)_{i},  & r \in N_{h,i},\\
\dps \sum_{j \in \Nui{i}} \bral - {\bf u_{h,i} \cdot  \nu_i}  + 
\alpha_{ij} \lambda_{h,i}  , 
\, \mu_{i} \brar_{ij} & = & 
\dps \sum_{j \in \Nui{i}} \bral {\bf u_{h,j} \cdot  \nu_j}  + 
\alpha_{ij} \lambda_{h,j}  , 
\, \mu_{i} \brar_{ij},  & \mu_{i} \in \Lambda_{h,i}.\\
\end{array}\end{equation}

It is proved that this problem (\ref{probtwd}) is well-posed in
\cite{MR98j:76083} and in \cite{vfcement-1999} for the finite volume method. It
is pointed out in \cite{MR98j:76083} that local mass conservation is insured
everywhere except on the interfaces. There is continuity of the normal
velocity ${\bf u \cdot \nu}$ across every element faces (or edges in 2D) that
does \emph{not lie on $\Sigma$}. This is the price to pay the non-matching meshes. The
flux is not conserved across any portion of $\Sigma$, but since $\mu_i = 1 \in
\Lambda_{h,i}$, we continue to conserve the mass globally.

A simple method to solve this problem (\ref{probtwd}) is to solve iteratively
the fixed point problem (\ref{probtwditer}). Given a $x^{0}_{h} \in
\Lambda_{h}$, where the superscript is used for the iteration $k \geq 0$ in the
method, solve iteratively

Compute $x^{k+1}_{h,i}|_{\Sigma_{ij}} =  
{\bf u^{k+1}_{h,i} \cdot \nu_i}  + \alpha_{ij} \lambda^{k+1}_{h,i} 
\quad j \in \Nui{i}$,  such that
 
$({\bf u^{k+1}_{h,i}}, p^{k+1}_{h,i}, \lambda^{k+1}_{h,i}) 
\in {\bf Z_{h,i}} \times N_{h,i} \times \Lambda_{h,i}$
is the solution of 
\begin{equation}\label{probtwditer}\begin{array}{rcll}
\left( K^{-1}{\bf u^{k+1}_h},\, {\bf v} \right)_{i} - 
\left( \div\; {\bf v}, \, p^{k+1}_h \right)_{i}  & = & \dps 
- \bral {\bf v \cdot \nu_i} , \, \lambda^{k+1}_{h,i} \brar_{i},  & {\bf v} 
\in {\bf Z_{h,i}}, \\
\left( \div\; {\bf u^{k+1}_h}, \, r \right)_{i} &= &
 \left(q_i, \, r \right)_{i},  & r \in N_{h,i},\\
\dps \sum_{j \in \Nui{i}} \bral - {\bf u^{k+1}_{h,i} \cdot \nu_i}  + 
\alpha_{ij} \lambda^{k+1}_{h,i}  , 
\, \mu_{i} \brar_{ij} & = & 
\dps \sum_{j \in \Nui{i}} \bral x^{k}_{h,i}  , 
\, \mu_{i} \brar_{ij},  & \mu_{i} \in \Lambda_{h,i}.\\
\end{array}\end{equation}

When the method has converged, the Robin interface conditions (\ref{interfrob})
are imposed weakly, and thus the solution of (\ref{probtwd}) is reached. It is
proved in \cite{vfcement-1999} that this fixed point method converges. One can
notice that it is non-symmetric eventhough the Laplace operator is symmetric.
It is advisable to accelerate the convergence with a non-symmetric Krylov
method, such as a BiCGStab or a GMRes method.

\subsection{Interface operators}

In order to clearly define the fixed point problem% and algorithm
, we introduce
discrete Robin to Robin operators ${S}_i$ in each subdomain $\Omega_i$ defined
by (\ref{steklovrob:def}).

\begin{equation}\label{steklovrob:def}\begin{array}{rlll}
{S}_i : \Lambda_{h,i} \times N_{h,i} & \longrightarrow & \Lambda_{h,i}, &\quad i \in I \\
{S}_i(x_{h,i} , q_i )|_{\Sigma_{ij}}  & = & 
{\bf u_{h,i} \cdot  \nu_i} + \alpha_{ij} p_{h,i},  &\quad i \in I, 
\quad \forall j \in \Nui{i},
\end{array}\end{equation}
such that ${\bf u_{h,i}}(x_h), p_{h,i}(x_h)$ is the solution of the following problem
(\ref{steklovrobw}) set on $\Omega_i, \ i \in I$.

We seek $({\bf u_{h,i}}, p_{h,i}, \lambda_{h,i}) \in {\bf Z_{h,i}} \times
N_{h,i} \times \Lambda_{h,i}$ such that
\begin{equation}\label{steklovrobw}\begin{array}{rcll}
\left( K^{-1}{\bf u_{h,i}},\, {\bf v} \right)_{i} - 
\left( \div\; {\bf v}, \, p_{h,i} \right)_{i}  & = & \dps 
- \bral {\bf v \cdot  \nu_i} , \, \lambda_{h,i} \brar_{i}, 
& {\bf v} \in {\bf Z_{h,i}}, \\
\left( \div\; {\bf u_{h,i}}, \, r \right)_{i} &= &
 \left(q_i, \, r \right)_{i},  & r \in N_{h,i},\\
\dps \sum_{j \in \Nui{i}} \bral - {\bf u_{h,i} \cdot  \nu_i}  
+ \alpha_{ij} \lambda_{h,i}  , 
\, \mu_{i} \brar_{ij} & = & 
\dps \sum_{j \in \Nui{i}} \bral x_{h,ij}  , 
\, \mu_{i} \brar_{ij},  & \mu_{i} \in \Lambda_{h,i}.\\
\end{array}\end{equation}

For a pair of neighbors $(\Omega_i, \Omega_j), \ (i,j) \in \Nui{}$, we define a
projection operator and a restriction operator in (\ref{e:Pij}) and in
(\ref{e:Rij}) that are defined on the interface between the subdomains. For a
subdomain $\Omega_i \ i \in I$, we define a restriction operator and a
projection operator in (\ref{e:Ri}) and in (\ref{e:Pi}) that are defined on the
internal boundary of the subdomain.

Let $P_{i\rightarrow j}$ be the $L^2$-projection operator onto $\tilde
\Lambda_{h,ij} = \Lambda_{h,ji}$. With this operator one can project the
$\lambda_{ij} \in \Lambda_{h,ij}$ onto $\tilde \Lambda_{h,ij}$.
\begin{equation}  \label{e:Pij}\begin{array}{rcll}
P_{i\rightarrow j} : L^2(\Sigma_{ij}) & \longrightarrow &  \tilde \Lambda_{h,ij}, \\
\dps  x_{ij} & \longmapsto & P_{i\rightarrow j} x_{ij} & \mbox{ such that } \\
\bral x_{ij} -  P_{i\rightarrow j} x_{ij}, \tilde \mu_{ij} \brar_{ij} 
& = & 0, & \quad \tilde \mu_{ij} \in  \tilde \Lambda_{h,ij}.
\end{array}\end{equation}


Define also the restriction operator from $\Lambda_{h,i}$ onto $\Lambda_{h,ij}$
\begin{equation} \label{e:Rij}\begin{array}{rcll}
  R_{ij}: \Lambda_{h,i} & \longrightarrow &  \Lambda_{h,ij}, \\
\lambda_i = \left(  \lambda_{ij_1}, 
      \lambda_{ij_2}, 
      \ldots,
      \lambda_{ij_{n_i}}
  \right)^{\top} 
 \mbox{ on } \Sh{i}  &  \longmapsto & \lambda_{ij} \mbox{ on } \Sh{ij}.
\end{array}\end{equation}

For a $i \in I$ we also define two restriction operators: one from
$\Lambda_{h}$ onto $\Lambda_{h,i}$, the other one from $\tilde \Lambda_{h}$
onto $\tilde \Lambda_{h,i}$.
\begin{equation} \label{e:Ri}\begin{array}{rcll}
  R_{i}: \Lambda_{h} & \longrightarrow &  \Lambda_{h,i}, \\
\lambda = \left(  \lambda_{1}, 
      \lambda_{2}, 
      \ldots,
      \lambda_{n}
  \right)^{\top} 
 \mbox{ on } \Sh{}  & \longmapsto  & \lambda_{i} \mbox{ on } \Sh{i}.\\
\tilde R_{i}: \tilde \Lambda_{h} & \longrightarrow &  \tilde \Lambda_{h,i}, \\
\tilde\lambda = \left( \tilde \lambda_{1}, 
     \tilde \lambda_{2}, 
     \ldots,
     \tilde \lambda_{n}
  \right)^{\top} 
 \mbox{ on } \tilde \Sh{}  & \longmapsto  & 
\tilde \lambda_{i} \mbox{ on } \tilde \Sh{i}.
\end{array}\end{equation}
By an abuse of notation, we use $s$ as a permutation from $\Lambda_{h}$ onto
$\tilde \Lambda_{h}$:
\begin{equation} \label{e:symm}\begin{array}{rcll}
s:\Lambda_{h} & \longrightarrow &  \tilde \Lambda_{h}, \\
s(\lambda_{h}) & = & \tilde \lambda_{h}.
\end{array}\end{equation}


Define also, for a $i \in I$, the projection operator from $L^2(\Sigma_i)$ onto
$\tilde \Lambda_{h,i}$. With this operator, one can project $\Lambda_{h,i}$
that lives on the {\it inner} boundary interface onto the {\it outer} boundary
interface $\tilde \Lambda_{h,i}$. 
\begin{equation} \label{e:Pi}\begin{array}{rcll}
  P_{i}: L^2(\Sigma_i) & \longrightarrow & 
\tilde \Lambda_{h,i},\\
\dps  x_{i} = \left(  x_{ij_1}, 
      x_{ij_2}, 
      \ldots,
      x_{ij_{n_i}}
  \right)^{\top} & \longmapsto & 
\left( P_{i\rightarrow j_1} x_{ij_1}, 
 P_{i\rightarrow j_2} x_{ij_2},
\ldots,
 P_{i\rightarrow j_n} x_{ij_n}
\right)^{\top}. \\
\end{array}\end{equation}

In all these notations, to solve the problem (\ref{probtwd}), it suffices to
find $\lambda_h \in \Lambda_h$ solution of the fixed point problem
(\ref{fixedpoint2}). The matrix-free matrix vector product that is necessary in
a Krylov method (such as BiCGStab) can be easily implemented in this form
(\ref{fixedpoint2}).
\begin{equation} \label{fixedpoint2}
  \mbox{Id}_{\Lambda_h} \lambda_{h} -
  s \left( \sum_{i \in I} \tilde R_i^{\top} P_i S_i (R_i \lambda_h, 0 ) \right) =
  s \left(\sum_{i \in I} \tilde R_i^{\top} P_i S_i ( 0, q|_{\Omega_i} ) \right) 
  \quad \mbox{in } \Lambda_h.
\end{equation}

\section{The \ocamlpiiil\ implementation}
%
% the coordination code is simple enough to be presented in extenso
% (comparer avec MPI?)

We studied so far a domain decomposition method for non-matching grids. To
perform the numerical tests, we programmed in {\tt C++} the subdomain solver
and added another layer to perform the coupling between the subdomains. We call
this a \emph{direct} simulation. In this section we present a way of coupling
the subdomain solver via an external program written in OCamlP3L. 
\subsection{Presentation}
We used the modules (Seqp3l, Parp3l, Grap3l) written in OCaml and denoted by
OCamlP3l, cf.~\cite{Ocamlp3lMlw98} and {\tt http://www.ocamlp3l.org/}.
OCamlP3l is a library that allows to write parallel programs using OCaml
according to skeleton models. This provides a very brief description of the
parallel structure of a program following the ideas of \cite{skeletoncole89}.
At first our main goal was just to {\it couple} some applications (written in
{\tt C, C++} or {\tt Fortran}) and the ability of OCamlP3L to couple {\it and
  parallelize} the applications appeared as an interesting extra feature to us.
It was particularly convenient that the parallel execution of our applications
required no extra development, once the sequential program is validated.

\subsection{Basic skeletons}

For the present study, we have used 4 of the basic skeletons provided by the
OCamlP3l environment.  The {\tt mapvector} skeleton encapsulates the parallel
execution of a skeleton over the components of a vector on a given number of
processes, typically the computation on each subdomain.  The {\tt loop}
skeleton is recursive, it encapsulates the repeated execution of a skeleton
until some condition is false, typically an iterative algorithm as BiCGStab to
solve a linear system.  The {\tt pipe} skeleton encapsulates the sequential
composition of two skeletons.  And the {\tt seq} skeleton encapsulates the
execution of a function in a sequential process.

The external encapsulation is performed by the function {\tt startstop}.
It encapsulates a skeleton by generating the input data and executing
post-processing, this defines an OCamlP3l network.

The {\tt pardo} function is used to call the (parallel) execution of a valid
OCamlP3l network defined by a {\tt startstop} call.

\subsection{Compilation}

To compile a network, one can use the tool {\tt ocampl3lcc} with one
of the 3 options {\tt -seq} (for sequential), {\tt -par} (for
parallel) or {\tt -gra} (for graphics). The purpose of the sequential
compilation is to run the program sequentially on one machine, or to
test the logic of the implemented algorithm with all the usual
sequential debugging tools. The graphics mode allows to get a picture
of the processor network described by the {\tt P3l} skeleton expression,
and it is useful to grasp the parallel structure of the program. The
parallel option allows to run in parallel the program, after debugging
in sequential and a simple recompilation.

\subsubsection*{OCaml coupling program}

We briefly describe the main program of the coupling code defined with the
function {\tt startstop} taking three input arguments. We want to point out
that we wrote a program based on a very simple algorithm and the resulting
coupling code is very short (about 170 lines), thanks to the conciseness of
the OCaml language and of the OCamlP3L environment.

\begin{figure}[htb]
  \begin{center} 
%    \includegraphics[width=6.2cm]{figures/RESU1}
%    \includegraphics[width=6.2cm]{figures/RESU10} \\
%    \includegraphics[width=6.2cm]{figures/RESU40}
%    \includegraphics[width=6.2cm]{figures/speed-up-eng} 
\end{center}
    \caption{A computation with 25 subdomains (the subdomains are
    square and have their own color). Initial solution (up left),
    solution computed after 10~iterations (up right) and final
    solution after 40~iterations (down left). The analytic solution is
    depicted in red.  Down right: speed up for 10 processors with
    refined grids: from 100$\times$100 cells up to 1400$\times$1400
    cells.}  \label{f:conv}
\end{figure}

The first argument is used to initialize the coupling. It generates
the boundary conditions for each subdomain. The second argument
finalizes the computations and prints the results. The third argument
describes the structure of the coupling. The algorithm that makes the
coupling is defined here. 

For these first experiments we chose a very simple example of overlapping
domain decomposition fixed point algorithm, using Dirichlet boundary
conditions, with matching and structured grids (Schwarz algorithm, see for
instance \cite{MR1857663} or \cite{MR95b:65147}). We do not discuss this
algorithm in this paper, but we simply remark that it is particularly easy to
program on a structured grid and thus was a good starting point to test the
coupling through OCamlP3L. Obviously it is not the algorithm that we intend to
use to make some more physical simulations.

The structure of the third argument is a loop. At each iteration, the program
solves on {\tt nbproc} processes a subdomain problem and returns as output a
vector of interface conditions. These interface conditions are taken as input
of a projection function {\tt proj} from a subdomain to its neighbor and then
the current result is plotted with {\tt plot}.
\begin{verbatim}
let program () = 
  startstop
    (generate_input_stream, ignore)
    (print_result, ignore, ignore)
    (loop (convergence, mapvector (seq (computation_on_a_subdomain), nbproc)
      ||| seq (proj) ||| seq (plot)))
in
pardo program;;
\end{verbatim}

\subsubsection*{Subdomain solver program (in C)}

The program that solves a subdomain problem has the following structure. It
reads on the standard input the vector containing the input Dirichlet boundary
conditions, solves the Poisson problem, and then returns on the standard output
the vector containing the computed output vectors. Then it waits for another
input vector.
The standard input and
output are redirected to the ones of the coupling program.

\subsection{Results}
\label{ss:dd-not}


The tests that are presented were performed on the unit square
$\Omega=[0,1]\times[0,1]$, divided into 5$\times$5 square subdomains with
regular matching grids. The overlap is made up of 2 cells. The evolution of the
solution for different iterations in the fixed point algorithm is presented in
Figure~\ref{f:conv}.

In Figure~\ref{f:conv} (down right), we can notice that we obtained a
speed up of order 7 on 10 processors. This is a reasonable results
considering that the implementation is not optimised at all. The speed
up seem unsensitive to the number of cells. This means that we did not
detected in this experiment the drawback of this simple algorithm: the
projection phase is performed with only one processor and can produce
delays due to communication overhead.

\section{Evaluation of the results}
%
% flexibility of the parfun and colors, speedup


\section{Conclusions and Future work}
%
% real world testing: oil
% auto-couplage

We presented in detail a nonoverlapping domain decomposition method for non
matching meshes based on Robin interface conditions. The matrix-free matrix
vector product can be easily implemented with the details provided. 
Two 2D numerical experiments illustrate the good
behavior of the method, particularly when the grids are nested, but not
necessarily. They also show that it can be applied to simulate the flow around
a nuclear waste disposal. Some more realistic 3D experiments are under way.

We also presented preliminary results of code coupling using the OCamlP3L
environment written in the OCaml language. The final goal is to couple codes
that solve different physics, or the same problem on different subdomains. The
first tests we performed are based on the very simple overlapping Schwarz
domain decomposition algorithm. The coupling consists in linking the subdomain
solvers. The results are produced with a particularly short coupling program.
The program can be executed in parallel, thus taking advantage of the natural
parallel properties of the domain decomposition method, with no extra
development cost: it only requires a simple recompilation. All this seems very
promising.

What we are working on now is the implementation of the 3D domain decomposition
method with non matching grids in the coupling code. Comparisons will be made
between the direct simulation and the computation using code coupling with
OCamlP3L, in terms of efficiency in solving the global problem, and in terms of
time required to develop the programs.


% in conclusions\ldots{}
% stream manipulation inside user functions

\bibliographystyle{alpha}

\bibliography{jfp}
% *****EXPORTBEGS: \bibliography{main.bbl}
\newcommand{\etalchar}[1]{$^{#1}$}
\begin{thebibliography}{DDCLP98}

\bibitem[Col89]{cole-th}
M.~Cole.
\newblock {\em {Algorithmic Skeletons: Structured Management of Parallel
  Computations}}.
\newblock Research Monographs in Parallel and Distributed Computing. Pitman,
  1989.

\bibitem[DDCLP98]{Ocamlp3lMlw98}
Marco Danelutto, Roberto Di~Cosmo, Xavier Leroy, and Susanna Pelagatti.
\newblock Parallel functional programming with skeletons: the ocamlp3l
  experiment.
\newblock {\em The ML Workshop}, 1998.

\bibitem[DFH{\etalchar{+}}93]{ic-parle-93-1}
J.~Darlington, A.~J. Field, P.~G. Harrison, P.~H.~J. Kelly, D.~W.~N. Sharp, and
  Q.~Wu.
\newblock {Parallel Programming Using Skeleton Functions}.
\newblock In {\em PARLE'93}, pages 146--160. Springer, 1993.
\newblock LNCS No. 694.

\bibitem[DGTY95]{darli-to-1}
J.~Darlington, Y.~Guo, H.~W. To, and J.~Yang.
\newblock {Parallel Skeletons for Structured Composition}.
\newblock In {\em Fifth ACM SIGPLAN Symposium on Principles and Practice of
  Parallel Programming}. ACM Press, July 1995.

\bibitem[DMO{\etalchar{+}}92]{fgcs-firenze}
M.~Danelutto, R.~Di Meglio, S.~Orlando, S.~Pelagatti, and M.~Vanneschi.
\newblock A methodology for the development and support of massively parallel
  programs.
\newblock {\em Future Generation Computer Systems}, 8(1--3):205--220, July
  1992.

\end{thebibliography}
