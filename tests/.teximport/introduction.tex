In a skeleton based parallel programming model 
\cite{cole-th,ic-parle-93-1,fgcs-firenze} a set of \textit{skeletons},
i.e. of second order functionals modelling common parallelism
exploitation patterns are provided to the user/programmer. The
programmer must use the skeletons to give parallel structure to
an application and  uses a plain sequential language to
express the sequential portions of the parallel application as
parameters to the skeletons. He/she has no other way to express
parallel activities but skeletons: no explicit process creation,
scheduling, termination, no communication promimitives, no shared
memory, no notion of being executing a program onto a parallel
architecture at all.

\ocamlpiiil\ is a programming environment that allows to write parallel
programs in \ocaml\footnote{See URL
  \texttt{http://pauillac.inria.fr/ocaml/}} according to the skeleton
model supported by the parallel language \pppl\footnote{See URL
  \texttt{http://www.di.unipi.it/.susanna/p3l.html}}, provides
seamless integration of parallel programming and functional
programming and advanced features like sequential logical debugging
(i.e. functional debugging of a parallel program via execution of the
architecture at allparallel code onto a
sequential machine) of parallel programs and strong typing, useful
both in teaching parallel programming and in building of
full-scale applications\footnote{See URL
  \texttt{http://qui.di.unipi.it/ocamlp3l.html} you will find relevant
  information, up to date references, documentation, examples,
  distribution code and dynamic web pages showcasing the \ocamlpiiil\
  features.}.
